\documentclass{beamer}
\usepackage{listings}
\usepackage{tabulary}


\mode<presentation>
{
  %\usetheme{Warsaw}
  % or ...

  \setbeamercovered{transparent}
  % \setbeamercovered{invisible}
  % or whatever (possibly just delete it)
}


\usepackage[english]{babel}
% or whatever

\usepackage[latin1]{inputenc}
% or whatever

% \usepackage{times}
% \usepackage[T1]{fontenc}
% Or whatever. Note that the encoding and the font should match. If T1
% does not look nice, try deleting the line with the fontenc.


% \title[Fortran \& Slide Rules] % (optional, use only with long paper titles)
\title
{Fortran \& Slide Rules}

\subtitle
{Elegant weapons for a more civilized age} % (optional)

\author[Mitton] % (optional, use only with lots of authors)
{kmitton}
% - Use the \inst{?} command only if the authors have different
%   affiliation.

\institute[Yelp] % (optional, but mostly needed)
{
  Yelp
}
% - Use the \inst command only if there are several affiliations.
% - Keep it simple, no one is interested in your street address.

\date[Engineering Learning Group 2015] % (optional)
{Engineering Learning Group / 2015-11-06}

\subject{Talks}
% This is only inserted into the PDF information catalog. Can be left
% out. 



% If you have a file called "university-logo-filename.xxx", where xxx
% is a graphic format that can be processed by latex or pdflatex,
% resp., then you can add a logo as follows:

% \pgfdeclareimage[height=0.5cm]{university-logo}{university-logo-filename}
% \logo{\pgfuseimage{university-logo}}



% Delete this, if you do not want the table of contents to pop up at
% the beginning of each subsection:
%\AtBeginSubsection[]
%{
%  \begin{frame}<beamer>{Outline}
%    \tableofcontents[currentsection,currentsubsection]
%  \end{frame}
%}


% If you wish to uncover everything in a step-wise fashion, uncomment
% the following command: 

%\beamerdefaultoverlayspecification{<+->}


\begin{document}

\begin{frame}
  \titlepage
\end{frame}

% \begin{frame}{Outline}
%   \tableofcontents
  % You might wish to add the option [pausesections]
% \end{frame}


\section{Fortran}

\subsection{Computers were big}

% \begin{frame}{Computers were big.}
%     % http://www.columbia.edu/cu/computinghistory/704-llnl.jpg
%     \includegraphics{images/columbia_704-llnl}
% \end{frame}

\begin{frame}[plain]
    \begin{centering}
        % http://www.columbia.edu/cu/computinghistory/704-llnl.jpg
        \pgfimage[height=\paperheight]{images/columbia_704-llnl}
        \par
    \end{centering}
\end{frame}


\begin{frame}[plain]
    \begin{centering}
        % http://www.fortran.com/ibm3.jpg
        \pgfimage[width=\paperwidth]{images/fortran_ibm3}
        \par
    \end{centering}
\end{frame}

\begin{frame}[plain]
    \begin{centering}
        % http://commons.wikimedia.org/wiki/File:Punch_card_Fortran_Uni_Stuttgart_(6).jpg
        % https://upload.wikimedia.org/wikipedia/commons/7/76/Punch_card_Fortran_Uni_Stuttgart_%286%29.jpg
        \pgfimage[width=\paperwidth]{images/wikimedia_punchcard}
        \par
    \end{centering}
\end{frame}


\begin{frame}[fragile]
\lstset{language=Fortran}
    \begin{lstlisting}
C=====================================
C      Computational subroutine 
       subroutine convec(x, y, z, nx, ny) 
       complex*16 x(*), y(*), z(*) 
       integer nx, ny
C      Initialize the output array. 
       do 10 i=1,nx+ny-1
          z(i) = (0.0,0.0) 
 10    continue
       do 30 i=1,nx 
          do 20 j=1,ny
             z(i+j-1) = z(i+j-1) + x(i) * y(j) 
 20       continue
 30    continue 
       return 
       end
    \end{lstlisting}
\end{frame}

\begin{frame}[fragile]
    \begin{tabulary}{\textwidth}{|L|L|L|}
        \hline
        1954-56 & Fortran & 32 statements including arithmetic \texttt{IF}: \newline 
            \texttt{IF (expr) label1, label2, label3} \\
        \hline
        1958 & Fortran II & Subroutines, Functions \\
        \hline
        1961-66 & Fortran IV / 66 & More data types, modern \texttt{IF} \\
        \hline
        1978 & Fortran 77 & \texttt{CHARACTER} type \\
        \hline
        1991-97 & Fortran 90/95 & Free-format source, Modules, \texttt{POINTER}, dynamic memory allocation \\
        \hline
        2003-08 & Fortran 2003 / 2008 & Objects, procedure pointers, I/O enhancements \\
        \hline
    \end{tabulary}
\end{frame}

\begin{frame}{Make Titles Informative. Use Uppercase Letters.}{Subtitles are optional.}
  % - A title should summarize the slide in an understandable fashion
  %   for anyone how does not follow everything on the slide itself.

  \begin{itemize}
  \item
    Use \texttt{itemize} a lot.
  \item
    Use very short sentences or short phrases.
  \end{itemize}
\end{frame}

\begin{frame}{Make Titles Informative.}

  You can create overlays\dots
  \begin{itemize}
  \item using the \texttt{pause} command:
    \begin{itemize}
    \item
      First item.
      \pause
    \item    
      Second item.
    \end{itemize}
  \item
    using overlay specifications:
    \begin{itemize}
    \item<3->
      First item.
    \item<4->
      Second item.
    \end{itemize}
  \item
    using the general \texttt{uncover} command:
    \begin{itemize}
      \uncover<5->{\item
        First item.}
      \uncover<6->{\item
        Second item.}
    \end{itemize}
  \end{itemize}
\end{frame}


\subsection{Second Subsection}

\begin{frame}{Make Titles Informative.}
\end{frame}

\begin{frame}{Make Titles Informative.}
\end{frame}



\section*{Summary}

\begin{frame}{Summary}

  % Keep the summary *very short*.
  \begin{itemize}
  \item
    The \alert{first main message} of your talk in one or two lines.
  \item
    The \alert{second main message} of your talk in one or two lines.
  \item
    Perhaps a \alert{third message}, but not more than that.
  \end{itemize}
  
  % The following outlook is optional.
  \vskip0pt plus.5fill
  \begin{itemize}
  \item
    Outlook
    \begin{itemize}
    \item
      Something you haven't solved.
    \item
      Something else you haven't solved.
    \end{itemize}
  \end{itemize}
\end{frame}


\end{document}


